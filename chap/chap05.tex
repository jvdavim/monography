\chapter{Avaliação Experimental}
% Explicar objetivos da avaliação experimental...
\section{Ambiente computacional}
% Máquinas utilizadas para os experimentos...

% Ambiente de programação (jupyter notebooks)...

% Bibliotecas (statmodels, fbprophet, wisardpkg etc)...

\section{Coleções de dados}
% Dados do covid19...

% Dados gerado de forma sintética com o ARMA com auxílio da biblioteca statmodels...
\section{Métricas de avaliação}
% Erro absoluto médio...

% Erro absoluto médio percentual...

% Raiz do erro médio quadrático...

% Validação cruzada adaptada para séries temporais...

\section{Resultados}
% Apresentar formato de apresentação dos resultados...
\subsection{Experimentos com dados sintéticos}
% Resultados com ARMA...

% Resultados com ARIMA...

% Resultados com Prophet...

% Resultados com AutoRegression WiSARD...

% Obs.: Apresentar escolha de hiperparâmetros, métricas de validação e tempos de execução.

\subsection{Experimentos com dados do covid}
% Resultados com ARMA...

% Resultados com ARIMA...

% Resultados com Prophet...

% Resultados com AutoRegression WiSARD...

% Obs.: Apresentar escolha de hiperparâmetros, métricas de validação e tempos de execução.

\subsection{Discussão}

