\chapter{Previsão de séries temporais com Regression WiSARD}
Como previsto no capítulo de redes neurais sem peso, o modelo Regression WiSARD é utilizado para resolver problemas de regressão, porém não é possível realizar previsão de séries temporais por depender de características que não são função do tempo. É possível, entretanto, tratar um problema de previsão de séries temporais como um problema de regressão aplicando técnicas como a janela deslizante e médias móveis conforme detalhado nas seções seguintes.

\section{Janelas deslizantes}
O método de janelas deslizantes é essencial para a Regression WiSARD ser capaz de realizar previsões de séries temporais. Isso ocorre porque é a técnica que permite transformar o problema temporal em um problema de regressão supervisionado.

% O método consiste em tornar cada amostra dependente das N amostras anteriores, e faz isso alocando uma janela de N amostras no início da série temporal e deslocando até o final de amostra em amostra para formar a matriz de características, conforme ilustra a Figure~\ref{fig:sliding_window}.

    % \begin{figure}[!ht] \label{fig:sliding_window}
    % \centering
    % \includegraphics[width=5.0in]{img/}
    % \caption{Legenda}
    % \end{figure}
% Exemplos práticos de utilização...

% Como ajuda na previsão de séries temporais com a Regression WiSARD...
\section{Média móvel}


% Exemplos práticos de utilização...

% Como ajuda na previsão de séries temporais com a Regression WiSARD...

\section{Regression WiSARD}
% Explicar treinamento do modelo...

% Explicar predições com o modelo...

% Vantagens - Aprendizado em tempo real, ...
