\chapter{Previsão de Séries Temporais}

\section{Definição do problema}
Previsão de séries temporais é uma solução adotada para muitos problemas da vida real com o objetivo de, a partir de dados observados, realizar extrapolações para o futuro. Esse tipo de abordagem é muito eficaz quando há poucos dados disponíveis dentro do escopo do problema, pois permite realizar inferências que utilizam dados históricos da própria variável, e podem ser essenciais para o auxílio à tomada de decisão.

% formalização.

% principais dificuldades.

\section{Modelos autoregressivos}
Existem muitos modelos que são frequentemente utilizados para tarefas de predição de séries temporais. Dentre estes, podemos citar desde modelos estatísticos como ARMA \cite{box&jenkins} e suas variações \cite{ALZAHRANI2020914} até métodos com modelos aditivos como o Prophet \cite{fbprophet}. Além desses, existem muitos estudos que fazem adaptações de Redes Neurais para esta tarefa como em \cite{ZHANG2005501} e até mesmo modelos misturados de Redes Neurais com ARIMA como apresentado em \cite{ZHANG2003159}.

Os modelos utilizados para os experimentos deste trabalho foram escolhidos com base em estudos paralelos realizados com os dados do Covid-19, e serão descritos nas subseções seguintes com suas respectivas formalizações.
\subsection{ARMA}
O modelo ARMA é uma combinação dos modelos auto-regressivo (AR) e de médias móveis (MA). O modelo auto regressivo puro de ordem $p$, ou seja, $AR(p)$, pode ser formalizado como:

\[X_{t}=c+\sum ^{p}_{i=1}\rho _{i}X_{t-i}+\varepsilon _{t}\]

onde $\rho _{1}\ldots \rho _{p}$ são os parâmetros, $c$ é uma constante e $\varepsilon _{t}$ representa um ruído branco.

O ajuste de parâmetros a ser realizado na etapa de treinamento do modelo pode ser feito utilizando o método dos mínimos quadrados ou método dos momentos através das equações de Yule-Walker.

O modelo de médias móveis (MA) é utilizado junto com o modelo auto regressivo, e é uma abordagem comum para modelagem de séries temporais univariadas. Este modelo estabelece uma dependência linear entre os valores atuais da série temporal e os valores passados, e pode ser formalizado como:

\[X_{t}=\mu+\varepsilon_{t}+\theta_{1} \varepsilon_{t-1}+\cdots+\theta_{q} \varepsilon_{t-q}\]

onde $\mu$ é a média da série, $\theta_{1}, ..., \theta_{q}$ são os parâmetros e $\varepsilon_{t}, ..., \varepsilon_{t-q}$ são os termos do ruído branco.

Cada modelo pode ser utilizado de forma independente, ou em conjunto formando o modelo ARMA. A construção desse modelo se dá pela soma do modelo auto-regressivo com o modelo de médias móveis, e pode ser formalizado como:

\[X_{t}=c+\varepsilon_{t}+\sum_{i=1}^{p} \varphi_{i} X_{t-i}+\sum_{i=1}^{q} \theta_{i} \varepsilon_{t-i}\]

Existem diversas técnicas para a otimização dos hiperparâmetros $p$ e $q$ para que o modelo seja melhor ajustado aos dados. Uma das formas mais aceitas para encontrar esses valores é o critério de Akaike, conforme recomendado por Peter J. Brockwell e Richard A. Davis em \cite{akaike}.

\subsection{ARIMA}
Alguns problemas de previsão de séries temporais são ligeiramente mais complexos, pois representam uma série temporal não estacionária. Nesse caso específico, o ajuste do modelo ARMA acaba não sendo bom, pois não acompanha a não estacionaridade dos dados. O modelo auto-regressivo integrado de médias móveis (ARIMA) se propõe a resolver o problema da estacionaridade, e pode ser estudado como uma generalização do modelo ARMA.

De forma análoga ao ARMA, os modelos ARIMA são geralmente denotados como $ARIMA(p, d, q)$, onde os parâmetros $p$, $d$ e $q$ são inteiros não negativos que representam a ordem do modelo auto-regressivo (AR), o grau de diferenciação da parte integrada (I) e a ordem do modelo de médias móveis (MA) respectivamente.

A parte integrada do modelo consiste na computação da diferença dos valores atuais da série temporal com os valores subsequentes da mesma. Esse procedimento é realizado $d$ vezes, e este é o parâmetro da parte integrada do modelo.

Pode-se formalizar o moldeo ARIMA como:

\[\left(1-\sum_{i=1}^{p} \phi_{i} L^{i}\right)(1-L)^{d} X_{t}=\left(1+\sum_{i=1}^{q} \theta_{i} L^{i}\right) \varepsilon_{t}\]

em que $X_{t}$ é a série temporal de dados, $t$ é um índice representado por um inteiro, $L$ é o operador de defasagem, $p$ é o hiperparâmetro do modelo auto-regressivo, $q$ é o hiperparâmetro do modelo de médias móveis, $d$ é o número de diferenças da parte integrada, $\phi_{i}$ são os parâmetros do modelo auto-regressivo, $\theta_{i}$ são os parâmetros do modelo de médias móveis e $\epsilon_{t}$ é o ruído branco ou erro aleatório.

A estimação de modelos ARIMA é realizada, geralmente, através do método de Box-Jenkins \cite{box&jenkins}, que é um processo iterativo para encontrar os parâmetros que melhor ajusta o modelo aos dados observados.

\subsection{Biblioteca statsmodels}
% apresentar implementação dos modelos autoregressivos.


\section{Modelo de previsão Prophet}
% explicar modelo aditivo baseado em tendência, sazonalidade e feriados/eventos.
\cite{fbprophet}
\subsection{Modelo de tendência}
% explicar modelo de tendência.
\subsection{Sazonalidade}
% explicar modelo de sazonalidade.
\subsection{Feriados e eventos}
% explicar influência de feriados e eventos no prophet.
\subsection{Biblioteca fbprophet}
% apresentar implementação do modelo de previsão prophet. 