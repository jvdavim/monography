\chapter{Previsão de Séries Temporais}

\section{Definição do problema}
Previsão de séries temporais é uma solução adotada para muitos problemas da vida real com o objetivo de, a partir de dados observados, realizar extrapolações para o futuro. Esse tipo de abordagem é muito eficaz quando há poucos dados disponíveis dentro do escopo do problema, pois permite realizar inferências que utilizam dados históricos da própria variável, e podem ser essenciais para o auxílio à tomada de decisão.

% Formalização...m

% Principais dificuldades..."editor.wordWrap": "on",

\section{Modelos}
Existem muitos modelos que são frequentemente utilizados para tarefas de predição de séries temporais. Dentre estes, podemos citar desde modelos estatísticos como ARMA \cite{box&jenkins} e suas variações \cite{ALZAHRANI2020914} até métodos com modelos aditivos como o Prophet \cite{fbprophet}. Além desses, existem muitos estudos que fazem adaptações de Redes Neurais para esta tarefa como em \cite{ZHANG2005501} e até mesmo modelos misturados de Redes Neurais com ARIMA como apresentado em \cite{ZHANG2003159}.

Os modelos utilizados para os experimentos deste trabalho foram escolhidos com base em estudos paralelos realizados com os dados do Covid-19, e serão descritos nas subseções seguintes com suas respectivas formalizações.
\subsection{ARMA}
O modelo ARMA é uma combinação dos modelos auto-regressivo (AR) e de médias móveis (MA). O modelo auto regressivo puro de ordem $p$, ou seja, $AR(p)$, pode ser formalizado como:

\[X_{t}=c+\sum ^{p}_{i=1}\rho _{i}X_{t-i}+\varepsilon _{t}\]

onde $\rho _{1}\ldots \rho _{p}$ são os parâmetros, $c$ é uma constante e $\varepsilon _{t}$ representa um ruído branco.

O ajuste de parâmetros a ser realizado na etapa de treinamento do modelo pode ser feito utilizando o método dos mínimos quadrados ou método dos momentos através das equações de Yule-Walker.

O modelo de médias móveis (MA) é utilizado junto com o modelo auto regressivo, e é uma abordagem comum para modelagem de séries temporais univariadas. Este modelo estabelece uma dependência linear entre os valores atuais da série temporal e os valores passados, e pode ser formalizado como:

\[X_{t}=\mu+\varepsilon_{t}+\theta_{1} \varepsilon_{t-1}+\cdots+\theta_{q} \varepsilon_{t-q}\]

onde $\mu$ é a média da série, $\theta_{1}, ..., \theta_{q}$ são os parâmetros e $\varepsilon_{t}, ..., \varepsilon_{t-q}$ são os termos do ruído branco.

Cada modelo pode ser utilizado de forma independente, ou em conjunto formando o modelo ARMA. A construção desse modelo se dá pela soma do modelo auto-regressivo com o modelo de médias móveis, e pode ser formalizado como:

\[X_{t}=c+\varepsilon_{t}+\sum_{i=1}^{p} \varphi_{i} X_{t-i}+\sum_{i=1}^{q} \theta_{i} \varepsilon_{t-i}\]

% Explicar possível utilização de variáveis exógenas...
\subsection{ARIMA}
Alguns problemas de previsão de séries temporais são ligeiramente mais complexos, 

% Definição formal do ARIMA...

% Expansões do ARIMA...

\section{Tecnologias}
% "Custurinha". Falar o que vimos até agora e o que vamos ver nas próximas subseções.
\subsection{Biblioteca statsmodels}
% Descrição da biblioteca (Linguagens de programação compatíveis, onde está disponível etc)...

% Visão breve das ferramentas proporcionadas pela biblioteca (ARMA, ARIMA etc)...

% Previsões de séries temporais com statmodels...
\subsection{Biblioteca fbprophet}
\cite{fbprophet}
% Previsão de séries temporais para negócios...

% Decomposição das séries temporais (tendência, sazonalidade e feriados)...

% Previsões de séries temporais com fbprophet...