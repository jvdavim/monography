\chapter{Introdução}

Com o início da pandemia da \textit{Coronavirus Disease} (COVID-19) em 2019, surgiram problemas causados pelo descontrole da taxa de transmissão do vírus na população e, portanto, aumentou-se a demanda de soluções para auxílio logístico às medidas médicas, sociais e econômicas. No Brasil, um dos problemas que se destacou no início da pandemia foi o de previsão do número de pessoas infectadas para os próximos dias em cada Estado, que é essencial para a tomada de decisão quanto às medidas preventivas.

No começo da pandemia, a baixa disponibilidade de dados no país foi um desafio que levou ao uso de estratégias alternativas, como o estudo do comportamento das séries temporais de contágio em pandemias passadas ou do cenário atual em países onde o contágio começou antes, como China, Itália, Alemanha, França, entre outros. Porém, com a disseminação do vírus, a busca de padrões na curva de infectados começou a se tornar viável, possibilitando caracterizar o problema como um problema de previsão de série temporal univariado.

Assim como a predição do número de infectados, existem outros problemas semelhantes de previsão de séries temporais como previsões meteorológicas, financeiras, de vendas, entre outros. Tais previsões se tornam cada vez mais relevantes, como mostrado em \citeauthor{DBLP:journals/corr/abs-1906-05433}, \citeyear{DBLP:journals/corr/abs-1906-05433} \cite{DBLP:journals/corr/abs-1906-05433} ao citar inúmeras vezes a importância fundamental de previsões acuradas à curto e longo prazo nos esforços relacionados à mitigação das mudanças climáticas. Portanto, generalizando o problema de previsão de séries temporais, e com a objetivo de propor uma alternativa de baixo custo computacional, baixo tempo de resposta e boa acurácia, este trabalho apresenta uma forma de resolver o problema utilizando Redes Neurais sem Peso e técnicas de pré-processamento que transformam o problema de previsão de série temporal em um problema de regressão supervisionado.

As Redes Neurais sem Peso são modelos que apresentam, em geral, tempo de inferência e treinamento menor que a maioria dos modelos como apresentando em \citeauthor{advanceswns}, \citeyear{advanceswns} \cite{advanceswns}, e são excelentes para tarefas que precisam de aprendizado em tempo real. Além disso, o uso de recursos computacionais é baixo, permitindo a sua execução em ambientes computacionais limitados, e sua utilização pode viabilizar aplicações em que o número de amostras é reduzido. Todas essas vantagens são consequência de sua arquitetura baseada em memória, que, agrupadas, formam discriminadores capazes de caracterizar uma classe em um problema de classificação supervisionado, ou de armazenar valores para reconhecimento de padrões em variáveis contínuas em um problema de regressão.

Como cada série temporal possui suas próprias características, suas componentes podem ser muito diferentes de caso para caso, e o sucesso de um modelo na tarefa de predição de uma série específica, pode não ser reprodutível em outras situações. Portanto, como extensão do objetivo deste trabalho, os experimentos são executados em três conjuntos de dados de origens diferentes, com a finalidade de avaliar o desempenho dos modelos de forma mais genérica dentro do problema de previsão de série temporal. Além do número de casos confirmados de COVID-19 no Rio de Janeiro, também são utilizados os dados de temperatura mínima diária em Melbourne, e dados gerados sinteticamente adicionando componentes criados de forma determinística.

A estrutura deste trabalho é a seguinte: o Capítulo~\ref{chap:02} apresenta o problema de previsão de séries temporais e os modelos autorregressivos (ARIMA) e aditivos (Prophet) como duas possíveis soluções, detalhando suas características e apresentando suas implementações. O Capítulo~\ref{chap:03}, por sua vez, introduz o conceito de Redes Neurais sem Peso e seus estimadores de regressão e classificação para problemas supervisionados através do detalhamento do modelo WiSARD e sua adaptação Regression WiSARD. Já as técnicas de pré-processamento para transformação do problema de previsão de séries temporais em um problema de regressão supervisionado são apresentadas no Capítulo~\ref{chap:04}, que, junto com o modelo Regression WiSARD, resolvem o problema de previsão de séries temporais. Além disso, também é apresentado um \textit{pipeline} completo de todas as transformações necessárias no pré-processamento para treinamento e inferência do modelo.

O ambiente computacional, as conjuntos de dados e as métricas de avaliação utilizadas são assuntos abordados no Capítulo~\ref{chap:05}, e são responsáveis por expor a metodologia do trabalho. Além disso, também são apresentados os resultados dos experimentos seguidos de uma breve discussão sobre os mesmos. Finalmente, o Capítulo~\ref{chap:06} recapitula o objetivo, o que foi apresentado, a conclusão dos resultados e os trabalhos futuros.