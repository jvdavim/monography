\chapter{Introdução}

Com o início da pandemia da \textit{Coronavirus Disease} (COVID-19) em 2019, surgiram problemas causados pelo descontrole da taxa de transmissão do vírus na população e, portanto, aumentou-se a demanda de soluções para auxílio logístico às medidas médicas, sociais e econômicas. No Brasil, um dos problemas que se destacou no início da pandemia foi o de previsão do número de pessoas infectadas para os próximos dias em cada Estado, que é essencial para a tomada de decisão quanto às medidas preventivas.

No começo da pandemia, a baixa disponibilidade de dados no país foi um desafio que levou ao uso de estratégias alternativas, como o estudo do comportamento das séries temporais de contágio em pandemias passadas ou do cenário atual em países onde o contágio começou antes como China, Itália, Alemanha, França, entre outros. Porém, com a disseminação do vírus, a busca de padrões na curva de infectados começou a se tornar viável, possibilitando caracterizar o problema como um problema de previsão de série temporal univariado.

Assim como o número de infectados, existem outros problemas semelhantes de previsão de séries temporais como previsões meteorológicas, financeiras, de vendas, entre outros. Portanto, generalizando o problema de previsão de séries temporais, e com a objetivo de propor uma alternativa de baixo custo computacional, baixo tempo de resposta e boa acurácia, este trabalho apresenta uma forma de resolver o problema utilizando Redes Neurais sem Peso e técnicas de pré-processamento que transformam o problema de previsão de série temporal em um problema de regressão supervisionado.

Para a avaliação e comparação dos resultados, o Capítulo~\ref{chap:02} apresenta o problema de previsão de séries temporais e os modelos autoregressivos (ARIMA) e aditivos (Prophet) como duas possíveis soluções, detalhando suas características e apresentando suas implementações. O Capítulo~\ref{chap:03}, por sua vez, introduz o conceito de Redes Neurais sem Peso e seus estimadores de regressão e classificação para problemas supervisionados através do detalhamento do modelo WiSARD (Wilkie, Stonhan and Aleksander Recognition Device) e sua adaptação Regression WiSARD. Já as técnicas de pré-processamento para transformação do problema de previsão de séries temporais em um problema de regressão supervisionado são apresentadas no Capítulo~\ref{chap:04}, que, junto com o modelo Regression WiSARD, resolvem o problema de previsão de séries temporais. Além disso, também é apresentado um \textit{piepline} completo de todas as transformações necessárias no pré-processamento para treinamento e inferência do modelo.