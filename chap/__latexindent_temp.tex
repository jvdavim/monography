\chapter{Previsão de Séries Temporais}

\section{Definição do problema}
Previsão de séries temporais é uma solução adotada para muitos problemas da vida real com o objetivo de, a partir de dados observados, realizar extrapolações para o futuro. Esse tipo de abordagem é muito eficaz quando há poucos dados disponíveis dentro do escopo do problema, pois permite realizar inferências que utilizam dados históricos da própria variável, e podem ser essenciais para o auxílio à tomada de decisão.

Toda previsão de série temporal possui um horizonte de previsão, ou seja, o número de instantes à frente que serão previstos. Esse valor faz parte da caracterização do problema, portanto, é fundamental que todos os ajustes sejam avaliados de acordo com esse horizonte.

Os modelos de previsão podem ser classificados em univariados ou multivariados. Modelos univariados utilizam apenas a própria variável como entrada para a previsão, enquanto modelos multivariados, podem utilizar multiplas variáveis para a previsão de uma variável alvo. Quando utilizada a própria variável para a inferência, esta é chamada de endógena, enquanto as restantes são chamadas exógenas.

Existem muitos modelos univariados que são frequentemente utilizados para tarefas de predição de séries temporais. Dentre estes, podem-se citar desde modelos estatísticos como ARMA \cite{box&jenkins} e suas variações \cite{ALZAHRANI2020914} até métodos com modelos aditivos como o Prophet \cite{fbprophet}. Além destes, existem muitos estudos que fazem adaptações de Redes Neurais para esta tarefa como em \cite{ZHANG2005501} e até mesmo modelos misturados de Redes Neurais com ARIMA como apresentado em \cite{ZHANG2003159}.

\section{Modelos autoregressivos}
Os modelos utilizados para os experimentos deste trabalho foram escolhidos com base em estudos paralelos realizados com os dados do Covid-19, e serão descritos nas subseções seguintes com suas respectivas formalizações.
\subsection{ARMA}
O modelo ARMA é uma combinação dos modelos auto-regressivo (AR) e de médias móveis (MA). O modelo auto regressivo puro de ordem $p$, ou seja, $AR(p)$, pode ser formalizado como:

\[X_{t}=c+\sum ^{p}_{i=1}\rho _{i}X_{t-i}+\varepsilon _{t}\]

onde $\rho _{1}\ldots \rho _{p}$ são os parâmetros, $c$ é uma constante e $\varepsilon _{t}$ representa um ruído branco.

O ajuste de parâmetros a ser realizado na etapa de treinamento do modelo pode ser feito utilizando o método dos mínimos quadrados ou método dos momentos através das equações de Yule-Walker.

O modelo de médias móveis (MA) é utilizado junto com o modelo auto regressivo, e é uma abordagem comum para modelagem de séries temporais univariadas. Este modelo estabelece uma dependência linear entre os valores atuais da série temporal e os valores passados, e pode ser formalizado como:

\[X_{t}=\mu+\varepsilon_{t}+\theta_{1} \varepsilon_{t-1}+\cdots+\theta_{q} \varepsilon_{t-q}\]

onde $\mu$ é a média da série, $\theta_{1}, ..., \theta_{q}$ são os parâmetros e $\varepsilon_{t}, ..., \varepsilon_{t-q}$ são os termos do ruído branco.

Cada modelo pode ser utilizado de forma independente, ou em conjunto formando o modelo ARMA. A construção desse modelo se dá pela soma do modelo auto-regressivo com o modelo de médias móveis, e pode ser formalizado como:

\[X_{t}=c+\varepsilon_{t}+\sum_{i=1}^{p} \varphi_{i} X_{t-i}+\sum_{i=1}^{q} \theta_{i} \varepsilon_{t-i}\]

Existem diversas técnicas para a otimização dos hiperparâmetros $p$ e $q$ para que o modelo seja melhor ajustado aos dados. Uma das formas mais aceitas para encontrar esses valores é o critério de Akaike, conforme recomendado por Peter J. Brockwell e Richard A. Davis em \cite{akaike}.

\subsection{ARIMA}
Alguns problemas de previsão de séries temporais são ligeiramente mais complexos, pois representam uma série temporal não estacionária. Nesse caso específico, o ajuste do modelo ARMA acaba não sendo bom, pois não acompanha a não estacionaridade dos dados. O modelo auto-regressivo integrado de médias móveis (ARIMA) se propõe a resolver o problema da estacionaridade, e pode ser estudado como uma generalização do modelo ARMA.

De forma análoga ao ARMA, os modelos ARIMA são geralmente denotados como $ARIMA(p, d, q)$, onde os parâmetros $p$, $d$ e $q$ são inteiros não negativos que representam a ordem do modelo auto-regressivo (AR), o grau de diferenciação da parte integrada (I) e a ordem do modelo de médias móveis (MA) respectivamente.

A parte integrada do modelo consiste na computação da diferença dos valores atuais da série temporal com os valores subsequentes da mesma. Esse procedimento é realizado $d$ vezes, e este é o parâmetro da parte integrada do modelo.

Pode-se formalizar o moldeo ARIMA como:

\[\left(1-\sum_{i=1}^{p} \phi_{i} L^{i}\right)(1-L)^{d} X_{t}=\left(1+\sum_{i=1}^{q} \theta_{i} L^{i}\right) \varepsilon_{t}\]

em que $X_{t}$ é a série temporal de dados, $t$ é um índice representado por um inteiro, $L$ é o operador de defasagem, $p$ é o hiperparâmetro do modelo auto-regressivo, $q$ é o hiperparâmetro do modelo de médias móveis, $d$ é o número de diferenças da parte integrada, $\phi_{i}$ são os parâmetros do modelo auto-regressivo, $\theta_{i}$ são os parâmetros do modelo de médias móveis e $\epsilon_{t}$ é o ruído branco ou erro aleatório.

A estimação de modelos ARIMA é realizada, geralmente, através do método de Box-Jenkins \cite{box&jenkins}, que é um processo iterativo para encontrar os parâmetros que melhor ajusta o modelo aos dados observados.

\subsection{Biblioteca statsmodels}
Os modelos autoregressivos apresentados, são implementados pela biblioteca statsmodels \cite{seabold2010statsmodels}, que é uma biblioteca para análise estatística e econométrica em Python. A biblioteca disponibiliza classes e funções para estimar diferentes modelos estatísticos, como ARMA e suas generalizações, além de testes estatísticos e análise estatística de dados.

O pacote possui a licensa de código aberto \textit{Modified BSD (3-clause)}, e sua documentação oficial fica hospedada em statsmodels.org.


\section{Modelo de previsão Prophet}
O modelo de previsão Prophet \cite{fbprophet} foi uma solução criada e adotada pelo Facebook para solução de problemas de previsão no nicho de negócios. A ideia era criar um modelo capaz de utilizar características comuns na maioria dos problemas de previsão para tentar melhorar a acertividade em relação à modelos considerados automáticos como o ARIMA. Então, para facilitar a utilização do modelo por analistas, que na maior parte dos casos, não possui conhecimento detalhado sobre a série temporal que está sendo analisada, o Prophet propõe um modelo que tenha hiperparâmetros intuitivos para alcançar maior escalabilidade no momento da otimização feita pelos analistas.

O modelo é resultado da decomposição da série temporal em 3 principais componentes: tendência, sazonalidade e feriados. Esses componentes compõe o modelo de acordo com a seguinte equação:

\[y(t) = g(t) + s(t) + h(t) + \epsilon_{t}\]

onde $g(t)$ representa a tendência, $s(t)$ a sazonalidade e $h(t)$ o feriado, e $\epsilon_{t}$ representa o erro não acomodado pelos outros modelos.

\subsection{Modelo de tendência}
Exitem 2 implementações do modelo de tendência que podem ser utilizados no Propet. Um é o modelo logístico de saturação, e o outro é o modelo linear.

O modelo de crescimento logístico é, inicialmente, dado por:

\[g(t)=\frac{C}{1+\exp (-k(t-m))}\]

onde $C$ é a capacidade de saturação, $k$ é a taxa de crescimento, e $m$ um parametro de deslocamento.

Porém, algumas características são ajustadas para atender melhor aos problema reais de previsão de séries temporais. A capacidade de saturação $C$, por exemplo, pode variar com relação ao tempo sendo substituída por $C(t)$. A taxa de crescimento $k$, também não é constante, então foi introduzido ao modelo um mecanismo de \textit{changepoints}, onde o usuário define um número específico de \textit{changepoints}, que podem ser escolhidos explicitamente ou automaticamente.

Os \textit{changepoints}, são pontos específicos na série temporal onde a taxa de crescimento $k$ pode ser alterada. Dada a sequência de \textit{changepoints} $s_{j}$, $j=1, \ldots, S$ onde $S$ é a quantidade de \textit{changepoints} escolhidos, é definido um vetor de ajuste das taxas $\boldsymbol{\delta} \in \mathbb{R}^{S}$. Cada elemento $\delta_{j}$ desse vetor é somado à taxa constante $k$ de crescimento ao alcançar o changepoint $j$ de forma acumulativa. Sendo assim, a taxa de crescimento pode ser calculada como $k+\mathbf{a}(t)^{\top} \boldsymbol{\delta}$, onde

\begin{equation}
    a_{j}(t)= \begin{cases}1, & \text { if } t \geq s_{j} \\ 0, & \text { otherwise. }\end{cases}
    \end{equation}

\subsection{Sazonalidade}
% explicar modelo de sazonalidade.
\subsection{Feriados e eventos}
% explicar influência de feriados e eventos no prophet.
\subsection{Biblioteca fbprophet}
% apresentar implementação do modelo de previsão prophet. 